\documentclass{mycv}
\usepackage[utf8]{inputenc}
\usepackage{float}
\usepackage[ngerman]{babel}
\usepackage{amsmath}
\usepackage{enumitem}		% Itemize
\usepackage{graphicx}
\usepackage{xcolor}
\usepackage{caption}
\usepackage{hologo}
\usepackage{hyperref}		% Links
\hypersetup{
	colorlinks,
	linkcolor={red!50!black},
	citecolor={blue!50!black},
	urlcolor={black!80!black}
}
\usepackage{bbding}			% Email symbol
\usepackage{fontawesome}	% Phone symbol
\usepackage[
top    = 2cm,
bottom = 3cm,
left   = 3.00cm,
right  = 2.50cm]{geometry}

%	Globally defines lists
\setlist[enumerate]{leftmargin=\the\myleng, rightmargin=0.5cm,  topsep=0.15cm, parsep=0cm, itemsep=0cm, label={\protect\raisebox{2pt}{\tiny\textbullet}}}
%	Margin to left, margin to top, margin between items, smaller textbullet

\begin{document}


%\vspace{1cm}
	
\cvtitle{Curriculum Vitae}
\name{Ulf Manhold}
\address{Lange Rötterstraße 110}{68167 Mannheim}
\contact{+49 157 3488 7818}{ulf\_manhold@gmx.de}
%\vspace{0.5cm}

\section{Ausbildung}
\periodentry{10/2015 -- 09/2018}{Ruprecht-Karls-Universität Heidelberg}
\studyentry{Master of Science Economics}{1.7}{The Labor Train in Vain: Quasi-Experimental Evidence on Eco-}{nomic Shocks and Migration}{Prof. Dr. Axel Dreher}
\periodentry{09/2011 -- 07/2015}{Universität Mannheim}
\studyentry{Bachelor of Science Volkswirtschaftslehre}{1.9}{Sequential Learning in a Population of Level-$k$ Players -- a Theo-}{retical Analysis}{Prof. Stefan P. Penczynski, Ph.D.}
\periodentry{09/2013 -- 01/2014}{Universidade Católica Portuguesa, Lissabon}
\erasentry{Erasmus Semester}{1.8}
\dateentry{06/2010}{Martin-Niemöller-Schule, Wiesbaden}
\abientry{Abitur}{1.7}{Englisch und Politik \& Wirtschaft}

\section{Praktische Erfahrungen}
\periodentry{10/2016 -- 10/2017}{Alfred Weber Institut, Universität Heidelberg}
\lprofentry{Wissenschaftliche Hilfskraft}{am Lehrstuhl Arbeitsmarkt-}{ökonomie und neue politsche Ökonomie von Prof. Christina}{Gathmann, Ph.D.}
\profitem{Mitarbeit bei verschiedenen Forschungsprojekten, Literaturrecherche, Aufbereitung und Analsye von Datensätzen}
\periodentry{01/2013 -- 04/2013}{Landesbank Hessen-Thüringen (Helaba), Frankfurt am Main}
\sprofentry{Praktikant}{in der Abteilung Volksiwrtschaft/Research}
\profitem{Betreuung des Projektes \textit{Die 100 größten Unternehmen in Hessen}, Recherche und Überprüfung des Datenmaterials, Aufbereitung und Analyse der Daten, weitere Recherchen zu versiedenen Themen}
\periodentry{01/2011 -- 07/2011}{Arbeiter-Samariter-Bund (ASB), Niedernhausen im Taunus}
\sprofentry{Zivildienstleistender}{im Bereich Hausnotrufservice}
\profitem{Durchführung der Hausnotrufaufgaben vor Ort und in der Zentrale, Dokumentation sowie Bearbeitung der Unterlagen für die erbrachten Leistungen}
\dateentry{12/2010}{Wiesbadener Kurier, Wiesbaden}
\sprofentry{Praktikant}{in der Lokalredaktion}
\profitem{Recherche und Verfassen von Reportagen und Vorberichten, Einblick in den gesamten Arbeitsprozess}


\section{Ehrenamtliches Engagement}
\periodentry{09/2012 -- 02/2018}{uni[ma]gazin e.V. -- unabhängiges Mannheimer Studierendenmagazin}
\profentry{Anzeigenmanagment, Ressortleiter und Autor}
\profitem{Akquise von Anzeigenkunden, Koordination und Redigat von veschiedenen Ressorts, Recherchen und Verfassen von Artikeln}
\profentry{1. Vorsitzender des Vereins}
\profitem{Leitung der Vereinssitzungen, Erstellen von Jahresberichten}

\section{Weiter Kenntnisse \& Qualifikationen}
\skill{Softwarekenntnisse}
\skillentry{Stata}{LaTeX}{Sehr gut}{MS Office}{Gut}{R}{Grundkenntnisse}
\skill{Sprachkenntnisse}
\langentry{Englisch}{Sehr gut}{Portugiesisch}{Gut}{Spanisch}{Grundkenntnisse}{Latinum}{}

\section{Sport}
\sinceentry{seit 2008}{Tennistrainer}
\profitem{Tennisschule Prätorius \& Pählich (bis 10/2017), Institut für Sport Uni Mannheim (2015--16), TC Rot-Weiss Waldpark Mannheim (seit 2018)}
\profitem{Ausbildung zum C-Trainer beim Badischen Tennis-Verband (2015)}


\signature{Unterschrift.jpg}
\append{Notenauszug Master, Bachelor- \& Abiturzeugnis, auf Anfrage Nachweise für Praktika, Zivildienst und Fremdsprachen (sofern vorhanden)}


\end{document} 